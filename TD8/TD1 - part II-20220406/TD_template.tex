\documentclass[11pt, a4paper]{article} % setsfont size and layout

%% required packages %%
\usepackage{mystyle} 
\usepackage[margin=2.5cm]{geometry} % margins

%% Here the main part of the document begins %%%

\begin{document}

%% Header on first page with course information etc. %%
\begin{tabular}{p{14.5cm}}
	{\large \textbf{ECO 102: Topics in Economics}} \\
	Ecole Polytechnique, Spring 2022  \\
	\textit{Professors}: Geoffroy Barrows, Benoit Schmutz \\
	\textit{Teaching Assistants}: Arnault Chatelain, Maddalena Conte \\
	\hline
	\\
\end{tabular}

\vspace*{0.3cm}	% vertical space between header on top of the page and main heading

\begin{center}
	{\Large \textbf{TD 1 (part 2): Maximisation and the social multiplier}}
	
\end{center} 

\vspace{0.4cm}

In this TD we will a review maximisation technique (the Lagrangian method) and apply it in the context of the social multiplier effect.

%%%%%%%%%%%%%%%%%%%%%%%%%%%%%%%%%%%%%%%%%%%%%%%%%%%%%%%%%%%%%%%%%%%%
\section*{Exercise}

\begin{enumerate}
    \item Review the Lagrangian method to solve for constrained optimization problems (we will see this together in class).
    \item Apply the Lagrangian method to solve for the following problem:
    
    Maximize \(f = x^{1/2} y^{1/2}\) subject to \(ax + cy = b \)
    
    Obtain first order conditions, \(x^{*}, y^{*}\) and \(\lambda^{*}\)
    
    \item Apply the Lagrangian method to the social multiplier effect seen in class (slide 11 of lecture 1). Show that:
    \begin{enumerate}
        \item \(U_x = p U_y\)
        \item \(\frac{dX}{dS} > 0 \Leftrightarrow U_{xs} > p U_{ys}\)
    \end{enumerate}
    \item (Extra) Interpret the lagrange multiplier \(\lambda\) in the general case seen in question 1.
    \item (Extra) Derive second-order conditions for constrained maximisation in the general case seen in question 1.
\end{enumerate}

%%%%%%%%%%%%%%%%%%%%%%%%%%%%%%%%%%%%%%%%%%%%%%%%%%%%%%%%%%%%%%%%%%%%
	
\end{document}


